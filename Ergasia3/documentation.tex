\documentclass{article}
\usepackage{graphicx}
\usepackage[english,greek]{babel}
\usepackage[utf8x]{inputenc}
\usepackage{listings}

\begin{document}

\title{\vspace{-3.5cm}\textbf{Υλοποίηση Συστημάτων Βάσεων Δεδομένων}\\Εργασία 3\\ Τεκμηρίωση Κώδικα}
\author{Λάμπρου Ιωάννης -\textlatin{1115201400088}\\ Χατζηδάκης Ραφαήλ -\textlatin{1115201400248}}

\maketitle
\paragraph{}
Για την εργασία αυτή, υλοποιήθηκαν πλήρως όλες οι συναρτήσεις επιπέδου SR που ζητήθηκαν στην εκφώνηση και συμπεριφέρονται σύμφωνα με αυτή. Όλες οι συναρτήσεις με εξαίρεση την \textlatin{SR\_SortedFile}, είναι βασισμένες στις συναρτήσεις της πρώτης άσκησης, με μικρές αλλαγές στα \textlatin{metadata}. \\ 
\paragraph{}
Στην συνάρτηση \textlatin{SR\_SortedFile}, για το πρώτο μέρος (ταξινόμηση ομάδων μπλόκ με αριθμό ίσο με το \textlatin{bufferSize}) χρησιμοποιήθηκε ο αλγόριθμος \textlatin{quicksort}, τροποποιημένος, βέβαια, για να εφαρμόζεται σε μια σειρά από μπλόκ. Οι ταξινομημένες ομάδες αυτές μπαίνουν σε ένα προσωρινό αρχείο με όνομα \textlatin{"temp"}. Έπειτα δημιουργούνται άλλα τόσα κενά μπλόκ στο αρχείο \textlatin{"temp"}. Αν το \textlatin{"input"} αρχείο έχει \textlatin{n}$-1$
μπλόκ, τότε και το \textlatin{"temp"} θα έχει $2*n-1$. Αυτό γίνεται για να υπάρχει αρκετός χώρος στο αρχείο για να αποθηκεύονται τα ενδιάμεσα αποτελέσματα του δεύτερου μέρους.
Για το δεύτερο μέρος, (κύριο μέρος του \textlatin{external sort}, κάθε φορά χρησιμοποιούνται για την σύγκριση τα πρώτα \textlatin{bufferSize}$-1$ μπλόκ και το τελευταίο χρησιμοποιείται ως \textlatin{output buffer} (για τα ενδιάμεσα αποτελέσματα στο \textlatin{"temp"} και τα τελικά αποτελέσματα).
Για τα τελικά αποτελέσματα, δημιουργείται ένα αρχείο με το δεδομένο όνομα, όπου και αποθηκεύονται, ενώ διαγράφεται το αρχείο \textlatin{"temp"}.

\paragraph{}
Η ανάπτυξη και εκτέλεση-δοκιμή της εργασίας έγινε σε περιβάλλον \textlatin{Ubuntu 16.04}.

\end{document}