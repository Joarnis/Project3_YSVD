\documentclass{article}
\usepackage{graphicx}
\usepackage[english,greek]{babel}
\usepackage[utf8x]{inputenc}
\usepackage{listings}

\begin{document}

\title{\vspace{-3.5cm}\textbf{Υλοποίηση Συστημάτων Βάσεων Δεδομένων}\\Εργασία 2\\ Τεκμηρίωση Κώδικα}
\author{Λάμπρου Ιωάννης -\textlatin{1115201400088}\\ Χατζηδάκης Ραφαήλ -\textlatin{1115201400248}}

\maketitle
\paragraph{}
Για την εργασία αυτή, υλοποιήθηκαν πλήρως όλες οι συναρτήσεις επιπέδου ΑΜ που ζητήθηκαν στην εκφώνηση και συμπεριφέρονται σύμφωνα με αυτή. \\ 
\paragraph{}
Κάθε αρχείο ευρετηρίου Β+ δέντρου αποτελείται από 3 είδη μπλοκ. Αρχικά, η ρίζα και τα κλαδιά του δέντρου είναι μπλοκ ευρετηρίου,
\textlatin{(index block)} έχοντας αρχικά \textlatin{".ib", 4 bytes}, έπειτα τον αριθμό των κλειδιών που είναι αποθηκευμένα σε αυτό και τέλος, δείκτη σε μπλοκ (αριθμός του μπλοκ) και κλειδί
τα οποία εναλλάσσονται ώσπου και τελειώνει σε δείκτη σε μπλοκ (ο αριθμός των δεικτών είναι κατά ένα μεγαλύτερος από τον αριθμό των κλειδιών).

Τα φύλλα του δέντρου είναι μπλοκ δεδομένων (περιέχουν εγγραφές) \textlatin{(data block)} αρχίζοντας από \textlatin{".db", 4 bytes}. Μέσα έχουν αποθηκεύμένο τον αριθμό των εγγραφών που υπάρχουν μέσα τους, έπειτα έναν δείκτη (αριθμό μπλοκ) στο επόμενο (πιο δεξιό) μπλοκ δεδομένων, ακολουθούμενο από τις εγγραφές που έχουμε εισάγει, 
(\textlatin{attribute1} και \textlatin{attribute2})

Τέλος, το πρώτο μπλοκ κάθε αρχείου είναι μπλοκ μεταδεδομένων και χρησιμοποιείται για να αποθηκεύει τον τύπο του αρχείου 
\textlatin{(index file (".if", 4 bytes)},τα \textlatin{attribute types, lengths} για κάθε έναν από τους τύπους δεδομένων που θα αποθηκεύονται στο αρχείο, καθώς και τον αριθμό του μπλοκ της ρίζας του Β+ δέντρου και τον αριθμό του πρώτου μπλοκ δεδομένων (αριστερότερο).

\paragraph{}
Για την υλοποίηση της άσκησης επίσης χρησιμοποιήθηκαν δύο πίνακες από \textlatin{extern structs}, οι \textlatin{Filemeta OpenIndexes, SearchData OpenSearches}, οι οποίες και χρησιμοποιούνται για την αποθήκευση χρήσιμων δεδομένων για τις \textlatin{Insert} και \textlatin{Scan} αντίστοιχα. Στην δομή  \textlatin{OpenIndexes} αποθηκεύεται ο
\textlatin{file descriptor} που δόθηκε από το σύστημα για την πρόσβαση σε αυτό, το όνομα του αρχείου, τα \textlatin{attribute types kai lengths} για κάθε έναν από τους τύπους δεδομένων που θα αποθηκεύονται στο αρχείο, καθώς και ο αριθμός του μπλοκ της ρίζας του Β+ δέντρου και ο αριθμός του πρώτου μπλοκ δεδομένων (αριστερότερο). 
Στην δομή \textlatin{OpenSearches} αποθηκεύεται η έξοδος της 
\textlatin{AM\_Find\_Next\_Entry},  
η θέση στον πίνακα \textlatin{OpenIndexes} του αρχείου στο οποίο γίνεται η αναζήτηση, η πράξη που γίνεται, το μπλοκ και η θέση στην οποία βρίσκεται η \textlatin{AM\_Find\_Next\_Entry} κατά τη διαδικασία της εκτύπωσης, και ένα κλειδί, ανάλογα με την πράξη.  
  
\paragraph{}
Για την υλοποίηση της \textlatin{Insert} σημειώνεται χρησιμοποιήθηκε μια αναδρομική συνάρτηση, η \textlatin{rec\_trav\_insert}, η οποία, αναδρομικά κατεβαίνει το Β+ δέντρο και όταν φτάσει σε φύλλο, \textlatin{(data block)} τότε και κάνει την εισαγωγή της δεδομένης εγγραφής, και, αν χρειαστεί να δημιουργηθεί κάποιο καινούργιο μπλοκ, τότε και περνάει τις απαραίτητες πληροφορίες σε ανώτερο επίπεδο του δέντρου με τιμή επιστροφής (στον εαυτό της από τον οποίο και κλήθηκε).

\paragraph{}
Ακόμα, για την ομάλή λειτουργία του \textlatin{AM\_Find\_Next\_Entry}, όταν δεν βρίσκεται κάποιο κλειδί, τότε και η τιμή του
\textlatin{OpenSearches[].curr\_pos} γίνεται ίση με -2, έτσι ώστε όταν αυτό εντοπιθεί από την \textlatin{AM\_Find\_Next\_Entry}, τότε και αυτή έχει ως έξοδο \textlatin{NULL} και αλλάζει τον κωδικό λάθους σε \textlatin{AME\_EOF}


\paragraph{}
Όλες οι λειτουργίες του ευρετηρίου (εισαγωγές ή εξαγωγές εγγραφών) δουλεύουν μια χαρά (\textlatin{(am\_main1, am\_main3)} αλλά φαίνεται, στην \textlatin{am\_main2} να μην μπορεί να διαβάσει από το μπλοκ των μεταδεδομένων, τον αριθμό του μπλοκ της ρίζας του δέντρου. Αυτό το πρόβλημα φαίνεται να λύνεται όμως, για κάποιο λόγο, αν αντιγράψουμε τον κώδικα της \textlatin{am\_main2} στην \textlatin{am\_main1}. 


\paragraph{}
Η ανάπτυξη και εκτέλεση-δοκιμή της εργασίας έγινε σε περιβάλλον \textlatin{Ubuntu 16.04}.

\end{document}